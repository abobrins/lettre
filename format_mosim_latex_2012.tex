%
\documentclass[a4paper,twocolumn,fleqn]{article}
\textwidth 17cm \textheight 247mm
\topmargin -4mm
\hoffset -9mm \voffset -14mm
%\setlength{\topmargin}{-0.8truecm}
\setlength{\oddsidemargin}{0.3cm}
\setlength{\evensidemargin}{0.3cm}
\setlength{\columnsep}{8mm}
\setlength{\parindent}{0mm}
\setlength{\parskip}{-2.0ex}
\setlength{\mathindent}{0mm}
\flushbottom


\usepackage{epsfig}
\usepackage{timesnew}
\usepackage{harvard}

\usepackage[latin1]{inputenc}
\usepackage{aeguill}
\usepackage[frenchb,english]{babel}
\usepackage{caption}


\setlength{\parskip}{2ex} \pagestyle{myheadings}
\markright{\hspace*{3.8cm} \textit{MOSIM'12 - June 06-08, 2012 - Bordeaux - France}}

\renewcommand{\thepage}{}
\renewcommand{\refname}{REFERENCES}



\makeatletter
\renewcommand\section{\@startsection{section}{1}{\z@}%
                       {-6\p@ \@plus -0\p@ \@minus -0\p@}%
                       {2\p@ \@plus 0\p@ \@minus 0\p@}%
                       {\normalsize\textbf}}

\renewcommand\subsection{\@startsection{subsection}{2}{\z@}%
                       {-6\p@ \@plus -0\p@ \@minus -0\p@}%
                       {2\p@ \@plus 0\p@ \@minus 0\p@}%
                       {\normalsize\textbf}}

\renewcommand\subsubsection{\@startsection{subsubsection}{3}{\z@}%
                       {-6\p@ \@plus -0\p@ \@minus -0\p@}%
                       {1\p@ \@plus 0\p@ \@minus 0\p@}%
                       {\normalsize\itshape\bfseries}}
\makeatother


\begin{document}

\title{ \vspace*{-25mm}
\centering
\fbox{\normalsize
\begin{minipage}{17cm}
\centering
\em \small{$9^{th}$ International Conference of Modeling, Optimization and Simulation - MOSIM'12 \\ June 6-8, 2012 - Bordeaux - France\\
"Performance, interoperability and safety for sustainable development"}
\end{minipage} }\\
\vspace*{6mm}
{\Large \textbf{PAPER TITLE}} {\normalsize (Times new roman, bold, 14 point type, above space 12 pts)}}

\author{
\begin{tabular}{cc}
\bf \normalsize {First A. AUTHOR, Second B. CO-AUTHOR} & \bf \normalsize{Third C. CO-AUTHOR} \\
\\
     \normalsize IMS  / University of Bordeaux & \normalsize CEREP ESSTT\\
     \normalsize 351, Cours de la Liberation & \normalsize 5 Avenue Taha Hussein, BP 56  \\
     \normalsize 33405 Talence cedex - France & \normalsize Bab Menara, 1008 Tunis - Tunisia\\
     \normalsize A.auteur@ims-bordeaux.fr, B.coauateur@ims-bordeaux.fr & \normalsize C.coauteur@planet.tn\\
     \\
\end{tabular}
}

\date{\begin{minipage}{17cm}
\normalsize
{\bf  ABSTRACT:}
\rm
{\em The abstract must contain no more than 200 words. The abstract must be justified on the page with left and right margins of 2 cm. The text of the abstract must be in italics, Times New Roman and 10 point type.}\\~\\
{\bf KEYWORDS:}
\rm
{\em 6 keywords maximum, italic, Times New Roman, 10 point type.}
\end{minipage}
}
\maketitle

\section{INTRODUCTION}

All manuscripts must be written in \textbf{English or in French}.

Leave 2.5 cm margins at both top and bottom of the page, 2 cm on both right and left sides. The space between the two columns should be 0.8 cm. Header and footer margins are 1.25 cm. Line spacing should be single line.

The whole paper should be written in "Times New Roman" font. Except paper title, the whole paper should be written in 10 fonts. Every paragraph should be justified. Leave one line space between two paragraphs. Paragraphs are not indented.

Paper title is in "Times New Roman" font, in 14-font size, in bold, centered, followed by two line spaces but must be preceded by no line space (distance between header and title is obtained by margin).

Authors' names and their affiliations must be defined as in presented example. The authors' names must be in bold and followed by a line space. It is necessary to put two line spaces between authors' affiliations and abstract.

Abstract must be followed by a white line. Keywords line is followed by two line spaces.


\section{HEADINGS}

Headings are in Times New Roman, in bold, in 10 fonts and capitalized. Leave one line above a major heading, and one line clear below before the start of the next paragraph or second-level heading.

\subsection{Subheadings (Second-Level Heading)}

Subheadings are in Times New Roman, in 10 point type, in bold. They are preceded and followed by a line space.

\subsubsection{Subsubheadings (Third-Level Heading)}

\vspace{-0.3cm}

Subsubheadings are in Times New Roman, in 10 point type, in bold and italic. They are preceded by a line space but are not followed by a line space.

A heading, whatever its level, must not appear alone at the bottom of a column.

\section{HEADERS}

The first page contains one header whose text is the following one, on the first line: \emph{$9^{th}$ International Conference of Modeling, Optimization and Simulation - MOSIM'12}, on the second line \emph{June 06-12, 2010 - Bordeaux - France}, and on the third line \emph{"Performance, interoperability and safety for sustainable development"}.

This header should be written in Times New Roman, in 10 fonts, centered, in italic and framed.

All following pages contain one header with the following text: \emph{MOSIM'10 - May 10-12, 2010 - Hammamet - Tunisia}.

These headers should be written in Times New Roman, in 10 fonts, centered and in italic.

\section{FIGURES AND TABLES}

All illustrations and graphs should be centered. Figures and Tables should be numbered separately and consecutively.

Figure and table captions should be flush center above the figures. Initially capitalize only the first word of each caption. Captions should be Times New Roman 10-point. Figure title must not be preceded by one line space. Figure title must be followed by one line space.

\begin{figure}[h]
%\centerline{\epsfig{file=figure.eps,height=6.2cm}}
\caption{Example of figure caption}
\vspace{-0.5cm}
\label{exfigure}
\end{figure}

The word "figure" which appears in the text of the article (making reference to presented figure) must be also put in its entirety (example: ... of the figure 1, it seems that the plan ...), and not in abbreviation (fig.).

\begin{table}[h]
\begin{center}
\begin{tabular}{|c|c|c|}
\hline
1 & 2 & 3 \\ \hline
2 & 3 & 4 \\ \hline
\end{tabular}
\caption{\label{''�tiquette''} Example of table caption}
\vspace{-0.5cm}
\end{center}
\end{table}

Figures and tables should be placed as close as possible to where they are cited. If it is quite necessary, figures and tables can be placed across two columns.

Photos will be in black and white.


\section{EQUATIONS}

The equation should be included in the text as on our example $x = \sum_{j=1}^N A_j$, or placed on separated lines:

\begin{equation}
  x = \sum_{j=1}^N A_j
\end{equation}

Equations can be numbered with equation numbers placed within parentheses and aligned against the right margin as shown in equation (1). Every equation appearing only on a line must be preceded and followed by a white line.

\section*{ACKNOWLEDGMENTS}

The titles "Acknowledgement" and "references" are not numbered.

\section*{REFERENCES}

In the text, references should be shown by the author's name, followed by the year in parentheses, e.g. (first author and second author, year). If more than two authors are involved, et al. should be used, e.g. (first author et al., year). If more than one paper by the same author appears in the same year, these should be distinguished by a and b, e.g. (first author and second author, 2000a), (first author and second author, 2001b).

List of references are given at the end of the article, arranged alphabetically according to first author, subsequent lines indented. Publications by the same author(s) should be listed in order of year of publication.

References should be given in the following format:

\bibliographystyle{dcu}
\begin{description}
  \item[] Dolgui A. and M.-A. Ould Louly, 2000a. An Inventory Control Model for MRP Parameterization. \textit{Second Conference IFAC on Management and Control of Production and Logistics (MCPL'2000)}, Grenoble, France, vol. 3, p. 1001-1006.
  \item[] Dolgui A. and M.-A. Ould Louly, 2000b. Optimization of Supply Chain Planning under Uncertainty. \textit{Preprints of the IFAC Symposium on Manufacturing, Modeling, Management and Control (MIM'2000)}, Patras, Greece, p. 291-296.
  \item[] Durand, J.A., 1996. \textit{Ordonnancement dynamique et r�actif dans les ateliers flow-shop hybrides : une approche � base d'algorithmes �volutionnistes hybrides}. Th�se de Doctorat, Universit� de Paris VI, France.
  \item[] Ellis, M., and B. Stroustrup, 1990. \textit{The annotated C++ reference manual}, Addison-Wesley.
  \item[] Houck, C. R., J. A. Joines, and M. G. Kay, 1996. Comparison of genetic algorithms, random restart, and two-opt switching for solving large location-allocation problems. \textit{Computers and Operations Research}, 23(6), p. 587-596.
  \item[] Joines, J.A. and C.R. Houck, 1992. Genetic Algorithm Optimization Toolbox for Matlab. Department of Industrial Engineering, Technical Rapport No. 92-01, North Carolina State University, Raleigh, North Carolina.
\end{description}

\newpage
\begin{minipage}{17cm}
\begin{center}
  \textbf{The previous text gives an example of template for definitive version of accepted articles.}
\end{center}

Full manuscript must be submitted in \textbf{English or in French} language. Authors are requested to provide a full paper of \textbf{10 pages} maximum, according to instructions provided within this template.\\

Full papers should be submitted in pdf format via the conference web site:

\begin{center}
\textbf{http://www.easychair.org/conferences/?conf=mosim12}
\end{center}

All submissions will be peer-reviewed by at least two experts and accepted papers will be published in proceedings with ISBN. Best papers will be proposed to journals (EJOR, IJPR, Simulation, EAAI, IJPE, JESA) for publication.

\vspace{.5cm}

\begin{center}
\textbf{Key dates}

\vspace{.5cm}
Submission of full papers: \textbf{November 14th, 2011}\\
Acceptance notification: \textbf{February 15th, 2012}\\
Submission of final papers: \textbf{April 1st, 2012}\\
\end{center}

To prepare at best conference proceedings, we thank you for sending final papers necessarily for April 1st, 2012 (papers received after this date will not appear in the conference preprints).
To be eligible for publication in the conference proceedings, an accepted paper must be presented at the conference by one of the authors.

\end{minipage}

\end{document}
